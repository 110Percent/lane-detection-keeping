\documentclass[titlepage]{article}


\usepackage[T1]{fontenc}
% The preceding line is only needed to identify funding in the first footnote. If that is unneeded, please comment it out.
\usepackage{cite}
\usepackage{amsmath,amssymb,amsfonts}
\usepackage{algorithmic}
\usepackage{blindtext}
\usepackage{booktabs}
\usepackage{graphicx}
\usepackage{indentfirst}
\usepackage{textcomp}
\usepackage{titlesec}
\usepackage{xcolor}
\graphicspath{ {./images/} }
\usepackage{geometry}
\geometry{
  a4paper,
  total={170mm,257mm},
  left=20mm,
  top=20mm,
}

\def\BibTeX{{\rm B\kern-.05em{\sc i\kern-.025em b}\kern-.08em
    T\kern-.1667em\lower.7ex\hbox{E}\kern-.125emX}}
\begin{document}

\title{A Lane Detection and Following System for Autonomous Vehicles\\
{\footnotesize Carleton University Engineering Capstone Group \#74}
}

\author{
	\begin{tabular}{cc}
		\begin{tabular}[t]{c}
			Curtis Davies      \\
			\texttt{101146353} \\
			\texttt{curtisrdavies@sce.carleton.ca}
		\end{tabular} &
		\begin{tabular}[t]{c}
			Liam Gaudet        \\
			\texttt{101155009} \\
			\texttt{liam.gaudet@carleton.ca}
		\end{tabular}       \\ \addlinespace[4ex]
		\begin{tabular}[t]{c}
			Ian Holmes         \\
			\texttt{101149794} \\
			\texttt{iana.holmes@carleton.ca}
		\end{tabular}       &
		\begin{tabular}[t]{c}
			Robert Simionescu  \\
			\texttt{101143542} \\
			\texttt{robert.simionescu@carleton.ca}
		\end{tabular}
	\end{tabular}
}



\maketitle
\tableofcontents


\newpage

\begin{abstract}
\end{abstract}



\section{Introduction}
As discussed in the above document, Chapter 1 is expected 
to be the standard introduction to the problem, concluding 
with an overview of the rest of the report.

\subsection{Identification of the Need}


\subsection{Definition of the Problem}


\subsubsection{Functional Requirements}


\subsubsection{Non-Functional Requirements}


\subsubsection{Constraints}


\subsection{Conceptual Solutions}

\subsubsection{Literary Review}

\subsubsection{Concepts}


\subsection{System Architecture}


\subsubsection{Software Architecture}


\subsubsection{Physical Architecture}

\subsection{Overview of Remainder of Report}

\section{The Engineering Project}

\subsection{Health and Safety}
Using the Health and Safety Guide posted on the course webpage, 
students will use this section to explain how they addressed 
the issues of safety and health in the system that they built 
for their project.

\subsection{Engineering Profesionalism}
Using their course experience of ECOR 4995 Professional Practice,
 students should demonstrate how their professional responsibilities 
 were met by the goals of their project and/or during the 
 performance of their project.

\subsection{Project Management}
One of the goals of the engineering project is real experience 
in working on a long-term team project. Students should explain 
what project management techniques or processes were used to 
coordinate, manage and perform their project.

\subsection{Justification of Suitability for Degree Program}
In this section, students should explain how the project 
relates to the degree program of each group member.
\subsection{Individual Contributions}
This section should carefully itemize the individual 
contributions of each team member. Project contributions 
should identify which components of work were done by each 
individual. Report contributions should list the author of 
each major section of this report.

\subsubsection{Project Contributions}
\subsubsection{Report Contributions}

\section{Work Plan}
\subsection{Work Breakdown Structure}
\subsection{Responsibility Matrix}
\subsection{Project Network}
\subsection{Gantt Chart}
\subsection{Costs, Special Components and Facilities}
\subsection{Risk Analysis}

\section{Subsystems}

\subsection{Lane Detection}

\subsubsection{Requirements}
\subsubsection{Technologies and Methods}
\subsubsection{Conceptualization}
\subsubsection{Software Architecture}
\subsubsection{Implementation}
\subsubsection{Evaluation}

\subsection{Lane Keeping \& Control}
\subsubsection{Requirements}
\subsubsection{Technologies and Methods}
\subsubsection{Conceptualization}
\subsubsection{Software Architecture}
\subsubsection{Implementation}
\subsubsection{Evaluation}


\section{System Integration and Evaluation}

\section{Reflections}

The final report needs to contain your original project proposal 
(for example, as an Appendix or a separate chapter in your main 
document). It is not uncommon that changes in your project goals 
and objectives, methods used to achieve them, etc., may have 
occurred over the course of the project. Therefore, in a final 
chapter in the report, entitled “Reflections”, discuss how well 
the original project objectives were met. Identify and discuss any
changes that occurred as the project progressed. Finally, as part 
of this chapter, reflect, as a group, on the past two terms. Did 
the project unfold as expected? Did the team work result in unexpected 
challenges or benefits? With hindsight, if you had to undertake the 
project again, would you make the same initial decisions about 
tools/methods/timelines?

\subsection{Success of Project Objectives}

\subsection{Changes from Proposal}

\subsection{Group Reflection}

\section*{References}

Please number citations consecutively within brackets \cite{b1}. The
sentence punctuation follows the bracket \cite{b2}. Refer simply to the reference
number, as in \cite{b3}---do not use ``Ref. \cite{b3}'' or ``reference \cite{b3}'' except at
the beginning of a sentence: ``Reference \cite{b3} was the first $\ldots$''

Number footnotes separately in superscripts. Place the actual footnote at
the bottom of the column in which it was cited. Do not put footnotes in the
abstract or reference list. Use letters for table footnotes.

Unless there are six authors or more give all authors' names; do not use
``et al.''. Papers that have not been published, even if they have been
submitted for publication, should be cited as ``unpublished'' \cite{b4}. Papers
that have been accepted for publication should be cited as ``in press'' \cite{b5}.
Capitalize only the first word in a paper title, except for proper nouns and
element symbols.

For papers published in translation journals, please give the English
citation first, followed by the original foreign-language citation \cite{b6}.

\begin{thebibliography}{00}
	\bibitem{b1} G. Eason, B. Noble, and I. N. Sneddon, ``On certain integrals of Lipschitz-Hankel type involving products of Bessel functions,'' Phil. Trans. Roy. Soc. London, vol. A247, pp. 529--551, April 1955.
	\bibitem{b2} J. Clerk Maxwell, A Treatise on Electricity and Magnetism, 3rd ed., vol. 2. Oxford: Clarendon, 1892, pp.68--73.
	\bibitem{b3} I. S. Jacobs and C. P. Bean, ``Fine particles, thin films and exchange anisotropy,'' in Magnetism, vol. III, G. T. Rado and H. Suhl, Eds. New York: Academic, 1963, pp. 271--350.
	\bibitem{b4} K. Elissa, ``Title of paper if known,'' unpublished.
	\bibitem{b5} R. Nicole, ``Title of paper with only first word capitalized,'' J. Name Stand. Abbrev., in press.
	\bibitem{b6} Y. Yorozu, M. Hirano, K. Oka, and Y. Tagawa, ``Electron spectroscopy studies on magneto-optical media and plastic substrate interface,'' IEEE Transl. J. Magn. Japan, vol. 2, pp. 740--741, August 1987 [Digests 9th Annual Conf. Magnetics Japan, p. 301, 1982].
	\bibitem{b7} M. Young, The Technical Writer's Handbook. Mill Valley, CA: University Science, 1989.
\end{thebibliography}
\vspace{12pt}
\color{red}
IEEE conference templates contain guidance text for composing and formatting conference papers. Please ensure that all template text is removed from your conference paper prior to submission to the conference. Failure to remove the template text from your paper may result in your paper not being published.


\appendix
\end{document}
